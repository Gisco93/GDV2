\newif\ifvimbug
\vimbugfalse

\ifvimbug
\begin{document}
\fi


\subsection{Bernstein-Bézier-Tensorprodukte (7 Punkte)}
\subsubsection{3 Punkte}
Wir wenden de Casteljau an mit $v=\frac{3}{4}$ und $\lambda= \frac{1}{4}$:\\
\begin{tikzpicture}[grow=left,
level 1/.style={sibling distance=15mm},edge from parent/.style={-,draw},>=latex, level 3/.style={edge from child/.style={->,draw},sibling distance=15mm}]

\node[root] {$\begin{pmatrix}-2.125\\ 2.5625\\5.4375\end{pmatrix}$}
     child {node[level 2] (c1) {$\begin{pmatrix}-1\\2\\3\end{pmatrix}$}
           child {node[level 2] (c11) {$\begin{pmatrix}-1\\2\\0\end{pmatrix}$}}
       child {node[level 2] (c21) {}}
       }
 child {node[level 2] (c2) {$\begin{pmatrix}-2.5\\2.75\\6.25\end{pmatrix}$}
	child {node[level 2] (c21) {$\begin{pmatrix}-1\\2\\4\end{pmatrix}$}}
       child {node[level 2] (c22) {$\begin{pmatrix}-3\\3\\7\end{pmatrix}$}}
       };
       \draw[-, draw] (c21) -- (c2);
\end{tikzpicture}\\

\begin{tikzpicture}[grow=left,
level 1/.style={sibling distance=15mm},edge from parent/.style={-,draw},>=latex, level 3/.style={edge from child/.style={->,draw},sibling distance=15mm}]

\node[root] {$\begin{pmatrix}3\\3.4375\\4.875 \end{pmatrix}$}
     child {node[level 2] (c1) {$\begin{pmatrix}3\\2.5\\3 \end{pmatrix}$}
           child {node[level 2] (c11) {$\begin{pmatrix}3\\1\\0\end{pmatrix}$}}
       child {node[level 2] (c21) {}}
       }
 child {node[level 2] (c2) {$\begin{pmatrix}3\\3.75\\5.5 \end{pmatrix}$}
	child {node[level 2] (c21) {$\begin{pmatrix}3\\3\\4\end{pmatrix}$}}
       child {node[level 2] (c22) {$\begin{pmatrix}3\\4\\6\end{pmatrix}$}}
       };
       \draw[-, draw] (c21) -- (c2);
\end{tikzpicture}\\

\begin{tikzpicture}[grow=left,
level 1/.style={sibling distance=15mm},edge from parent/.style={-,draw},>=latex, level 3/.style={edge from child/.style={->,draw},sibling distance=15mm}]

\node[root] {$\begin{pmatrix}6.5\\1.375\\6.5625 \end{pmatrix}$}
     child {node[level 2] (c1) {$\begin{pmatrix}7.25\\1.75\\3\end{pmatrix}$}
           child {node[level 2] (c11) {$\begin{pmatrix}8\\1\\0\end{pmatrix}$}}
       child {node[level 2] (c21) {}}
       }
 child {node[level 2] (c2) {$\begin{pmatrix}6.25\\1.25\\7.75\end{pmatrix}$}
	child {node[level 2] (c21) {$\begin{pmatrix}7\\2\\4\end{pmatrix}$}}
       child {node[level 2] (c22) {$\begin{pmatrix}6\\1\\9\end{pmatrix}$}}
       };
       \draw[-, draw] (c21) -- (c2);
\end{tikzpicture}\\
Wir wenden de Casteljau an mit $u=\frac{1}{4}$ und $\lambda= \frac{3}{4}$:\\
\begin{tikzpicture}[grow=left,
level 1/.style={sibling distance=15mm},edge from parent/.style={-,draw},>=latex, level 3/.style={edge from child/.style={->,draw},sibling distance=15mm}]

\node[root] {$\begin{pmatrix}0.3359375\\2.81640625\\5.296875   \end{pmatrix}$}
     child {node[level 2] (c1) {$\begin{pmatrix}-0.84375\\2.78125\\5.296875\end{pmatrix}$}
           child {node[level 2] (c11) {$\begin{pmatrix}-2.125\\ 2.5625\\5.4375\end{pmatrix}$}}
       child {node[level 2] (c21) {}}
       }
 child {node[level 2] (c2) {$\begin{pmatrix}3.875\\2.921875\\5.296875\end{pmatrix}$}
	child {node[level 2] (c21) {$\begin{pmatrix}3\\3.4375\\4.875\end{pmatrix}$}}
       child {node[level 2] (c22) {$\begin{pmatrix}6.5\\1.375\\6.5625 \end{pmatrix}$}}
       };
       \draw[-, draw] (c21) -- (c2);
\end{tikzpicture}\\
\includegraphics[width=0.9\textwidth]{2_a.png}
\subsubsection{2 Punkte}
hierzu bestimmen wir die kontrollpunkte an den Intervallsgrenzen, wie in Aufgabeteil a. in der grafik sind ist die Neue untereilung eingezeichnet.\\ 
\includegraphics[width=0.9\textwidth]{2_b.png}
\subsubsection{2 Punkte}
\includegraphics[width=0.5\textwidth]{2_c_P1.png}\\
\includegraphics[width=0.5\textwidth]{2_c_P2.png}\\
\includegraphics[width=0.5\textwidth]{2_c_P3.png}
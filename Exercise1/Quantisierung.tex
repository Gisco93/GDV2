\newif\ifvimbug
\vimbugfalse

\ifvimbug
\begin{document}
\fi


\subsection{Quantisierung von Positionsdaten (4 Punkte)}
\subsubsection{2 Punkte}
Quantiesierung für b=2. Erster Pfeil Quantisierung und zweiter Pfeil Dekomprimierung.\\
$\begin{pmatrix} 0\\1 \end{pmatrix} \rightarrow \begin{pmatrix} 2\\3 \end{pmatrix} \rightarrow \begin{pmatrix} 1/3\\1 \end{pmatrix}\\
\begin{pmatrix} -0.7\\-0.7 \end{pmatrix} \rightarrow \begin{pmatrix}0\\0 \end{pmatrix} \rightarrow \begin{pmatrix} -1\\-1 \end{pmatrix}\\
\begin{pmatrix} 0.7\\0.7 \end{pmatrix} \rightarrow \begin{pmatrix} 3\\3 \end{pmatrix} \rightarrow \begin{pmatrix} 1\\1 \end{pmatrix}\\
\begin{pmatrix} -1\\0 \end{pmatrix} \rightarrow \begin{pmatrix} 0\\2 \end{pmatrix} \rightarrow \begin{pmatrix} -1\\1/3 \end{pmatrix}\\
\begin{pmatrix} 1\\0 \end{pmatrix} \rightarrow \begin{pmatrix} 3\\2 \end{pmatrix} \rightarrow \begin{pmatrix} 1\\1/3 \end{pmatrix}\\
\begin{pmatrix} -0.7\\0.7 \end{pmatrix} \rightarrow \begin{pmatrix} 0\\3 \end{pmatrix} \rightarrow \begin{pmatrix} -1\\1 \end{pmatrix}\\
\begin{pmatrix} 0.7\\-0.7 \end{pmatrix} \rightarrow \begin{pmatrix} 3\\0 \end{pmatrix} \rightarrow \begin{pmatrix} 1\\-1 \end{pmatrix}\\
\begin{pmatrix} 0\\0 \end{pmatrix} \rightarrow \begin{pmatrix} 2\\2 \end{pmatrix} \rightarrow \begin{pmatrix} 1/3\\1/3 \end{pmatrix}\\
\begin{pmatrix} 0\\-1 \end{pmatrix} \rightarrow \begin{pmatrix} 2\\0 \end{pmatrix} \rightarrow \begin{pmatrix} 1/3\\-1 \end{pmatrix}\\$

Quantiesierung für b=3.\\
$\begin{pmatrix} 0\\1 \end{pmatrix} \rightarrow \begin{pmatrix}4\\7 \end{pmatrix} \rightarrow \begin{pmatrix} 1/7\\1 \end{pmatrix}\\
\begin{pmatrix} -0.7\\-0.7 \end{pmatrix} \rightarrow \begin{pmatrix}1\\1 \end{pmatrix} \rightarrow \begin{pmatrix} -5/7\\-5/7 \end{pmatrix}\\
\begin{pmatrix} 0.7\\0.7 \end{pmatrix} \rightarrow \begin{pmatrix} 6\\6 \end{pmatrix} \rightarrow \begin{pmatrix}5/7 \\5/7\end{pmatrix}\\
\begin{pmatrix} -1\\0 \end{pmatrix} \rightarrow \begin{pmatrix} 0\\4 \end{pmatrix} \rightarrow \begin{pmatrix} -1\\1/7 \end{pmatrix}\\
\begin{pmatrix} 1\\0 \end{pmatrix} \rightarrow \begin{pmatrix} 7\\4 \end{pmatrix} \rightarrow \begin{pmatrix} 1\\1/7 \end{pmatrix}\\
\begin{pmatrix} -0.7\\0.7 \end{pmatrix} \rightarrow \begin{pmatrix} 1\\6 \end{pmatrix} \rightarrow \begin{pmatrix} -5/7\\5/7 \end{pmatrix}\\
\begin{pmatrix} 0.7\\-0.7 \end{pmatrix} \rightarrow \begin{pmatrix} 6\\1 \end{pmatrix} \rightarrow \begin{pmatrix} 5/7\\-5/7 \end{pmatrix}\\
\begin{pmatrix} 0\\0 \end{pmatrix} \rightarrow \begin{pmatrix} 4\\4\end{pmatrix} \rightarrow \begin{pmatrix} 1/7\\1/7 \end{pmatrix}\\
\begin{pmatrix} 0\\-1 \end{pmatrix} \rightarrow \begin{pmatrix} 4\\0 \end{pmatrix} \rightarrow \begin{pmatrix} 1/7\\-1 \end{pmatrix}\\$

\subsubsection{2 Punkte}
Ein Punkt $p$ mit einer $AABB(m,M)$ und Bildtiefe $b$ auf den Punkt $p*$. Daraus folgt das es ein $q\in[0.2^b-1]$ gibt, sodass $dekomprimierung(q) = p_q = p^*$. Der maximale Fehler entsteht nun wenn ein Nachbarpunkt $p_{q+1}$ dekomprimiert wird. Der Abstand von $p_q$ un $p_{q+1}$ ist:\\
$$|p_{q+1}-p_q|_2 = |\frac{1}{2^b-2^0}(M-m)|_2$$
$p$ wird auf $q$ abgebildet wird, folgt das $p$ auf $p^*$ abgebildet damit gilt für den maximalen Fehler:\\
$$Fehler_{kompression} = | p - p^*|_2 \leq |\frac{1}{2} \frac{1}{2^b-2^0}(M-m)|_2 $$
Der mittlere Fehler ist dann die Hälfte davon.
\subsubsection{1 Punkt}
$$ 0.5 \leq \frac{1}{2} \frac{1}{2^b-2^0}\cdot5450 $$\\
$$ 1 \leq  \frac{1}{2^b-2^0}\cdot5450$$\\
$$ \frac{1}{5450} \leq  \frac{1}{2^b-2^0}$$\\
$$ 5450 \geq  2^b-1$$\\
$$ 5451 \geq  2^b$$\\
$$ \log(5451) \geq  \log(2)\cdot b$$\\
$$ \approx 12,45 \geq  b$$\\
Damit werden mindestens 13 Bit benötigt.


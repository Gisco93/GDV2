\newif\ifvimbug
\vimbugfalse

\ifvimbug
\begin{document}
\fi


\subsection{Emissions-Absorptions Model (6 Punkte)}
\begin{tabular}{lcr}
  Dichtewert & q & k \\
  1 & 0.1 & 0.1 \\
  2	& 0.4 & 0.2\\
  3	& 0.9 & 0.3\\
  5	& 2.5 & 0.5\\
  6	& 3.6 & 0.6\\
  7	& 4.9 & 0.7\\
 \end{tabular}

Bilineare Interpolation über Stützpunkte R1 und R2:\\
\\
$k(R_1) = \frac{4-3.5}{4-3} * 0.7 + \frac{3.5-3}{4-3} * 0.1 = 0.35 + 0.05 = 0.4$\\
$q(R_1) = 0.5 * 4.9 + 0.5 * 0.1 = 2.45 + 0.05 = 2.5$\\
$k(R_2) = 0.5 * 0.6 + 0.5 * 0.6 = 0.6$\\
$q(R_2) = 0.5 * 3.6 + 0.5 * 3.6 = 3.6$\\
$k(s_1) = \frac{3-2.75}{3-2} * k(R_1) + \frac{2.75 - 2}{3-2} * k(R_2) = 0.25 * 0.4 + 0.75 * 0.6 = 0.55$\\
$q(s_1) = 0.25 * 2.5 + 0.75 * 3.6 = 3.325$\\
\\
Also ergibt sich $I(s_{1})$ folgenderma"sen, wobei man das Delta X einfach an der X-achse ablesen kann(Delta x = 0.5):\\
\\
$I(s_1) = I(s_0) * e^{-k(s_1)*\Delta x} + q(s_1)*\Delta x = 0.5 * \frac{1}{e^{0.55*0.5}} + 3.325 * 0.5 = 2.042$\\
\\
Nach dem selben Prinzip wird der Rest auch berechnet.\\
\\
$q(R_3) = \frac{2-1.5}{2-1} * 0.4 + \frac{1.5-1}{2-1} * 0.9 = 0.2 + 0.45 = 0.65$\\
$k(R_3) = 0.5 * 0.2 + 0.5 * 0.3 = 0.1 + 0.15 = 0.25$\\
$q(R_4) = 0.5 * 3.6 + 0.5 * 4.9 = 4.25$\\
$k(R_4) = 0.5 * 0.6 + 0.5 * 0.7 = 0.65$\\
$k(s_2) = \frac{1}{4} * 0.25 + \frac{3}{4} * 0.65 = 0.55$\\
$q(s_2) = 0.25 * 0.65 + 0.75 * 4.25 = 3.35$\\
$I(s_2) = 2.042 * \frac{1}{e^{0.55*2}} + 3.35 * 2 = 7.379$\\
\\
Und für s3 braucht man den ersten Schritt nicht, da sich der Punkt auf der y-Achse zwischen zwei Gitterpunkten befindet:\\
\\
$k(s_3) = \frac{1}{4} * 0.5 + \frac{3}{4} * 0.2 = 0.275$\\
$q(s_3) = 0.25 * 2.5 + 0.75 * 0.4 = 0.925$\\
$I(s_3) = 7.779 * \frac{1}{e^{0.275*1.5}} + 0.925 * 1.5 = 6.272$\\
\\
Und weil es so schön war nocheinmal das ganze für $\hat{s}$....\\
\\
$q(R_5) = \frac{1}{2} * 0.9 + \frac{1}{2} * 2.5 = 1.7$\\
$k(R_5) = 0.5 * 0.3 + 0.5 * 0.5 = 0.4$\\
$q(R_6) = 0.5 * 4.9 + 0.5 * 0.1 = 2.5$\\
$k(R_6) = 0.5 * 0.7 + 0.5 * 0.1 = 0.4$\\
$k(\hat{s_1}) = \frac{1}{2} * 0.4 + \frac{1}{2} * 0.4 = 0.4$\\
$q(\hat{s_1}) = 0.5 * 1.7 + 0.5 * 2.5 = 2.1$\\
$I(\hat{s_1}) = 0.5 * \frac{1}{e^{0.4*0.5}} + 2.1 * 0.5 = 1.459$\\
\\
Für $\hat{s2}$:\\
\\
$q(R_7) = \frac{1}{2} * 0.1 + \frac{1}{2} * 0.4 = 0.25$\\
$k(R_7) = 0.5 * 0.1 + 0.5 * 0.2 = 0.15$\\
$q(R_8) = 0.5 * 0.4 + 0.5 * 0.9 = 0.65$\\
$k(R_8) = 0.5 * 0.2 + 0.5 * 0.3 = 0.25$\\
$k(\hat{s_2}) = \frac{1}{2} * 0.15 + \frac{1}{2} * 0.25 = 0.2$\\
$q(\hat{s_2}) = 0.5 * 0.25 + 0.5 * 0.65 = 0.45$\\
$I(\hat{s_2}) = 1.459 * \frac{1}{e^{0.2 * 2}} + 0.45 * 2 = 1.878$\\
\\
Und $\hat{s3}$:\\
\\
$k(\hat{s_3}) = 0.5$\\
$q(\hat{s_3}) = 0.5 * 2.5 + 0.5 * 2.5 = 2.5$\\
$I(\hat{s_3}) = 1.878 * \frac{1}{e^{0.5*1.5}} + 2.5 * 1.5 = 4.637$\\
